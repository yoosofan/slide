\documentclass[12pt, a4paper]{article}
\usepackage{setspace}
\usepackage{subfigure}
\usepackage{algorithm}
\usepackage{algorithmic}
\usepackage{listings}
\usepackage{graphicx}
\usepackage{multicol}
\usepackage{color}
\usepackage{amsmath}
%\usepackage[usenames,dvipsnames]{xcolor}
%\usepackage{backgroud}
\usepackage{geometry}
\geometry{top=15mm, bottom=20mm, left=10mm, right=10mm}
%\usepackage[utf8]{inputenc}
%\usepackage[utf8]{fontenc}
%\usepackage{arabtex}
%\usepackage[farsi,arabic]{babel}
%\usepackage[utf8]
\usepackage{array}
\usepackage{xepersian}
\settextfont[Scale=1.2]{Nazli}
%\settextfont[Scale=1.1]{Times New Roman}
\defpersianfont\nastaliq{IranNastaliq}
\setlatintextfont[Scale=1.1]{Times New Roman}
%\setlatintextfont[Scale=1.1]{Libertine}
%\setlatintextfont{Times}
\author{Ahmad Yoosofan}
\SepMark{-}
\date{\today}
\begin{document}
\pagestyle{empty}  % suppress page number
\lstset{% general command to set parameter(s)
basicstyle=\small, % print whole listing small
keywordstyle=\bfseries, % underlined bold black keywords
commentstyle=\fcolorbox{blue},
identifierstyle=, % nothing happens
language=C,
stringstyle=\ttfamily, % typewriter type for strings
showstringspaces=false,
numbers=left, numberstyle=\small, stepnumber=1, 
numbersep=5pt,
numberblanklines=true, %|false
%backgroundcolor=\color{yellow},
showstringspaces=false
} % no special string spaces
\centering{
\begin{nastaliq} به نام خدا \end{nastaliq}
\\
آزمون مبانی برنامه نویسی ، زمان 150 دقیقه
}
\\ \raggedleft
لطفا روی برگه‌ی آزمون، پاسخنامه و دیگر برگه‌ها نام خود را بنویسید.
 لطفا همه‌ی برگه‌های خود را پس از آزمون تحویل دهید. پاسخ‌گویی به پرسش‌ها در ۲۰ دقیقه‌ی نخست آزمون و سپس پس از هر نیم ساعت به مدت ۱۰ دقیقه خواهد بود ۴۵ دقیقه آخر آزمون به پرسشی پاسخ داده نمی‌شود.
 در حل مسأله‌های آزمون هیچ تابع ریاضی یا رشته‌ای آماده‌ای را به کار نبرید و تابع‌هایی را که نیاز دارید خودتان بنویسید
\begin{enumerate}
\item
خروجی برنامه زیر را بنویسید.(۱)

\begin{latin}
\lstinputlisting[multicols=3]{test1.c}
\end{latin}

\item
تابع مقایسه‌ی دو رشته را به صورت زیر بنویسید که دو رشته را دریافت کند و اگر یکمی
\lr{(str1)}
  کوچکتر از دومی
\lr{(str2)} 
بود مقدار منفی یک را برگرداند و اگر یکمی بزرگ‌تر از دومی بود مثبت یک را برگرداند و اگر دو رشته طول مساوی و نویسه‌های (کاراکترهای)
یکسان داشتند آن‌گاه صفر را برگرداند.(۱/۲۵)
\\
\begin{latin}
"Ab" <"Abc" = "Abc" < "abc" < "bac" = "bac" <"bacd" < "da" ="da"  \\
\begin{lstlisting}[numbers=none] 
int str_cmp(const char *str1 , const char *str2);
\end{lstlisting}
\end{latin}

\item
محاسبه ترکیب m و n را به صورت بازگشتی می‌توان به صورت
زیر
تعریف کرد. تابعی به صورت بازگشتی بنویسید که دو عدد صحیح m و n را بگیرد و بر پایه‌ی فرمول داده شده ترکیب m و n را حساب کند. فرض کنید همواره  
\lr{0 $\leq$ m  $\leq$ n} 
است.(۱/۲۵)
\\

\begin{latin}
\begin{math}
\binom{n}{m} = \binom{n-1}{m} + \binom{n-1}{m-1} \quad  ,  \quad  \binom{n}{0} =  1  \quad  ,  \quad  \binom{n}{n} =  1 
\end{math}
\end{latin}

\begin{latin}
\begin{lstlisting}[numbers=none] 
int cmb(int m, int n);
\end{lstlisting}
\end{latin}

\item
تابع جستجوی دودویی را بنویسید که یک آرایه‌ی مرتب شده و غیر تکراری(a) را به همراه تعداد عنصرهای آن(n) و عددی که باید
 جستجو شود(v) را می‌گیرد و مکان v را در آرایه برمی‌گرداند و اگر نیافت منفی یک را برمی‌گرداند.(۱)
\\
\begin{latin}
\begin{lstlisting}[numbers=none] 
int bin_search(int a[], int n, int v);
\end{lstlisting}
\end{latin}

\item
تابع
\lr{exp(x)}
را به کمک بسط سری 
\lr{\begin{math} \exp(x) = 1 + \frac{x}{1!} + \frac{x^2}{2!}+\frac{x^3}{3!}+ ... + \frac{x^n}{n!} \end{math} }
بنویسید. این تابع یک عدد اعشاری x را می‌گیرد و یک عدد اعشاری برمی‌گرداند که برابر
\lr{ \begin{math} e^x \end{math} }
است.
 محاسبه سری را تا جایی ادامه دهید که آخرین جمله آن
\lr{ \begin{math} \frac{x^n}{n!}\end{math} }
کوچکتر از 0/001 شود.(۱/۷۵)
\begin{latin}
\begin{lstlisting}[numbers=none] 
double exp(double x);
\end{lstlisting}
\end{latin}

\item
تابعی بنویسید که اشاره‌گرهایی به دو عدد بزرگ را دریافت کند و سپس این دو عدد بزرگ را با هم جمع کند و نتیجه را برگرداند. ساختار عدد صحیح بزرگ در زیر نوشته شده است.
این عددها بدون علامت هستند و همواره مثبت هستند. اگر به شکل دیگری نیز می‌توانید  تابع جمع دو عدد صحیح بزرگ را بنویسید.(۱/۷۵)

\begin{latin}
\begin{lstlisting}[numbers=none] 
typedef struct {int a[1000];int n;}hn; hn add(hn*a1, hn*a2);
\end{lstlisting}
\end{latin}
\end{enumerate}
تندرست باشید، احمد یوسفان
\end{document}
