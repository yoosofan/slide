\documentclass[12pt, a4paper]{article}
\usepackage{setspace}
\usepackage{subfigure}
\usepackage{algorithm}
\usepackage{algorithmic}
\usepackage{listings}
\usepackage{graphicx}
\usepackage{multicol}
\usepackage{color}
%~ %\usepackage[usenames,dvipsnames]{xcolor}
%~ %\usepackage{backgroud}
\usepackage[top=15mm, bottom=20mm, left=10mm, right=10mm]{geometry}
%~ %\usepackage[utf8]{inputenc}
%~ %\usepackage[utf8]{fontenc}
%~ %\usepackage{arabtex}
%~ %\usepackage[farsi,arabic]{babel}
%~ %\usepackage[utf8]
\usepackage{xepersian}
\usepackage{array}
\settextfont[Scale=1.2]{Nazli}
%~ %\settextfont[Scale=1.1]{Times New Roman}
\defpersianfont\nastaliq{IranNastaliq}
%~ \setlatintextfont[Scale=1.1]{Times New Roman}
%~ %\setlatintextfont[Scale=1.1]{Libertine}
%~ %\setlatintextfont{Times}
\author{Ahmad Yoosofan}
\SepMark{-}
\date{\today}
\begin{document}
\pagestyle{empty}  %~ suppress page number
\centering{\begin{nastaliq} به نام خدا \end{nastaliq}}
\\
\centering{
آزمون مبانی برنامه نویسی ، زمان ۱۱۰ دقیقه
}
\begin{enumerate}
\item
اشکال‌های برنامه‌ی زیر را بیابید(۲). 
%~ %\setlatinmonofont[Scale=1]{Courier}  
\\
%~ %\setlatintextfont{courier}
\begin{latin}
\lstset{%~ general command to set parameter(s)
basicstyle=\small, %~ print whole listing small
keywordstyle=\bfseries, %~ underlined bold black keywords
commentstyle=\fcolorbox{blue},
identifierstyle=, %~ nothing happens
language=C,
stringstyle=\ttfamily, %~ typewriter type for strings
showstringspaces=false,
numbers=left, numberstyle=\small, stepnumber=1, 
numbersep=5pt,
numberblanklines=true, %~ |false
%~ %backgroundcolor=\color{yellow},
showstringspaces=false
} %~ no special string spaces

%~ \begin{lstlisting}[frame=single,multicols=2]

%~ \begin{lstlisting}[][multicols=2]
\begin{lstlisting}  
#include<stdio.h>
int main(){
  int i,n,a[100];
  i=0;
  printf("Enter  0 < n < 100 : ");
  scanf("%d",& n);
  if(n>0){
    if(i<n){
      while(i<n){
        printf("Enter a[%d] : ", i);
        scanf("%d",& a[i]);
        i = i + 1;
      }
      sum = 0;
      while(i<n){
        sum = sum+a[i];
        i = i + 1 ;
      }
      printf("The output is: %d", sum);
    }
    else{
      printf("Wrong input! n>=100 \n");
    }
  }
  else{
    printf("Wrong input. n <=0 \n");
  }
  return 0;
}
\end{lstlisting}
\end{latin}
\item
برنامه‌ای بنویسید که یک عدد بزرگ‌تر از ۲ از کاربر بگیرد و یک مربع به اندازه‌ی آن به کمک * بکشد. در نوشتن برنامه‌ی این مسأله و مسأله‌های پس از این نیازی نیست که مانند برنامه‌ی بالا درست بودن ورودی کاربر  بررسی شود(۵). 
\begin{center} 
\begin{tabular}{| c   | c | c  | c | c| c |} \hline
   عدد  & 
شکل  & 
عدد    & 
شکل  & 
عدد      & 
شکل 
   \\ %~ \tabularnewline 
\hline
  &     &   &****  &   & ***** \\
  & *** &   &****  &   & ***** \\
3 & *** & 4 &****  & 5 & ***** \\
  & *** &   &****  &   & ***** \\
  &     &   &      &   & ***** \\
\hline \end{tabular} \end{center} 
\item
برنامه‌ای بنویسید که ۵ عدد نخست Emirp را نمایش دهد.
عددهای Emirp عددهایی هستند که اگر رقم‌های آنها را معکوس کنیم باز عدد اولی به دست می‌آید. برای نمونه ۱۳
یکی از این عددها است که معکوس رقم‌های آن یعنی ۳۱ نیز اول است(۹).

\item
برنامه‌ای بنویسید که یک عدد از کاربر بگیرد و  رقم‌های آن را از کوچک به بزرگ مرتب کند و عدد به دست آمده را نمایش دهد(۹).
\\
\begin{center} 
\begin{tabular}{| c   | c | c  | c | c| c |}
\hline
   عدد1  & 
خروجی1  & 
عدد۲    & 
خروجی۲  & 
عدد۳     & 
خروجی۳ 
    \tabularnewline 
\hline
 491 &  149   &  318 & 138  &  729  & 279 \\
\hline \end{tabular} \end{center}
\\ %~ [8mm]
تندرست باشید،‌احمد یوسفان
\end{enumerate}
\end{document}
